
\documentclass[12pt,t]{beamer}
\usepackage{graphicx}
\usepackage{fontawesome}
\setbeameroption{hide notes}
\setbeamertemplate{note page}[plain]
\graphicspath{{figures/}} % Location of the graphics files


\usepackage{helvet}

\usetheme{default}
\beamertemplatenavigationsymbolsempty

\definecolor{foreground}{RGB}{73,73,73}
\definecolor{background}{RGB}{247,247,247}
\definecolor{uwred}{RGB}{197, 5, 12}
\definecolor{title}{RGB}{197, 5, 12}
\definecolor{gray}{RGB}{155,155,155}
\definecolor{jhblue}{RGB}{0,38,73}

\definecolor{subtitle}{RGB}{40,70,75}
\definecolor{hilight}{RGB}{102,255,204}
\definecolor{vhilight}{RGB}{255,111,207}

\setbeamercolor{titlelike}{fg=title}
\setbeamercolor{subtitle}{fg=subtitle}
\setbeamercolor{institute}{fg=foreground}
\setbeamercolor{normal text}{fg=foreground,bg=background}

\setbeamercolor{item}{fg=foreground} % color of bullets
\setbeamercolor{subitem}{fg=gray}
\setbeamercolor{itemize/enumerate subbody}{fg=gray}
\setbeamertemplate{itemize subitem}{{\textendash}}
\setbeamerfont{itemize/enumerate subbody}{size=\footnotesize}
\setbeamerfont{itemize/enumerate subitem}{size=\footnotesize}

\setbeamertemplate{footline}{%
    \raisebox{5pt}{\makebox[\paperwidth]{\hfill\makebox[20pt]{\color{gray}
          \scriptsize\insertframenumber}}}\hspace*{5pt}}

\newcommand{\bi}{\begin{itemize}}
\newcommand{\ei}{\end{itemize}}
\newcommand{\ig}{\includegraphics}
\newcommand{\subt}[1]{{\footnotesize \color{subtitle} {#1}}}

{
\setbeamertemplate{footline}{} % no page number here
\fontfamily{cmss}\selectfont
\title{	UW-Madison Visit Day, 2018}
\subtitle{Department of Biostatistics and Medical Informatics}
\author{\href{https://jmmaronge.github.io/}{Jacob M. Maronge}}
\institute{\href{https://www.stat.wisc.edu/}{Department of Statistics \\ Department of Biostatistics and Medical Informatics} \\[4pt] \href{http://www.wisc.edu}{University of Wisconsin{\textendash}Madison}}
\date{\href{https://jmmaronge.github.io/}{\tt \scriptsize https://jmmaronge.github.io/}
\\[-4pt]
\href{https://github.com/jmmaronge}{\tt \scriptsize \faGithub \ jmmaronge}
\\[-4pt]
\href{https://twitter.com/jmmaronge}{\tt \scriptsize \faTwitter \ @jmmaronge}
}


\begin{document}

{
\setbeamertemplate{footline}{} % no page number here
\frame{
  \titlepage
  
}
}
%% normal frame
\begin{frame}{Introduction}
\subt{About me}
 
\bigskip
\begin{itemize}
\item 2nd year PhD student in statistics and biostatistics
\item Working with Professor Paul Rathouz
\item  Research interests: clinical biostatistics, experimental design and longitudinal or clustered data, including biased sampling schemes
\end{itemize}


\end{frame}

\begin{frame}{Goals}
\subt{Why am I here?}
 
\bigskip

I'd like to offer an example of some of the interesting work, \textbf{from a student prospective}, that you can be a part of here at UW.
\bigskip

Two Examples:
\begin{itemize}
\item Analysis of Speech Trajectories in Children with Cerebral Palsy (CP)
\item Power Analysis for Longitudinal Data Using Decomposition
\end{itemize}


\end{frame}

\begin{frame}{Analysis of Speech Trajectories}
\subt{in children with CP}
\bigskip

\begin{itemize}
\item Collaborators: Wisconsin Intelligibility, Speech, and Communication Laboratory (WISC Lab, PI: Professor Katie Hustad)
\item CP is an umbrella term for a group of permanent movement disorders, which can also affect speech
\item Clinicians categorize these children into 3 categories based off of speech abilities. 
\end{itemize}

\textbf{\color{uwred}Question:} Do data-driven methods regarding analysis of speech patterns collected on 35 children in the cohort suggested existence of these clinical categories? 

\end{frame}

\begin{frame}{Analysis of Speech Trajectories}
\begin{center}
\setlength{\fboxsep}{0pt}%
\setlength{\fboxrule}{1pt}%
\fcolorbox{subtitle}{background}{\ig[width=0.6\textwidth]{fitted_plots_for_speech_analysis_speech_for_handout.pdf}}
\end{center}
Subject level data for 4 speech variables with group level fitted values from a multivariate longitudinal model

\end{frame}

\begin{frame}{Power Analysis for Longitudinal Data}
\subt{Using Decomposition}
\bigskip

\begin{itemize}
\item Power analysis and sample size calculations are a crucial component of any study design.
\item Many study designs lead to either complicated formulas or do not have closed form solutions
\item We propose a method for calculating power in the situation of clustered or longitudinal data that leads to a solution that is both simple to calculate and gives valuable insight to how study design parameters affect power.
\end{itemize}
\end{frame}


\begin{frame}{Setup}
\subt{Model}
\bigskip

Consider the situation where predictors are randomly observed. Our model of interest is,
\begin{equation} \label{eq:model}
Y_{ij}=\beta_0+ \beta X_{ij}+ \epsilon_{ij},
\end{equation}
where $i=1,\ldots, m$ denotes the study participant and $j=1,\ldots,n$ denotes measurements within subject.
\begin{itemize}
\item corr$(X_{ij},X_{ij'}) = \tau_X >0$
\item corr$(\epsilon_{ij},\epsilon_{ij'}) = \tau_\epsilon>0$
\item $\mbox{Var}(X_{ij})=\mbox{Var}(\epsilon_{ij}) = 1$
\end{itemize}

\end{frame}

\begin{frame}{Decomposition}
\subt{Results}
\bigskip

By decomposing the model into pieces which vary within-subject and between-subject, we are able to derive the correlation between $X_{ij}$ and $Y_{ij}$, 
\begin{equation} \label{eq:rhoeff}
\rho^2_{\mbox{eff}} =\frac{\rho^2\gamma}{\rho^2\gamma+(1-\rho^2)}.
\end{equation}
Interesting pieces
\begin{itemize}
\item $\rho$ denotes the correlation as if there were no correlation within subject.
\item $\gamma$ is a function of  $\tau_X,  \tau_\epsilon$, and $n$ (study design parameters).
\end{itemize}
\end{frame}

\begin{frame}{Power Analysis for Longitudinal Data}
\begin{center}
\setlength{\fboxsep}{0pt}%
\setlength{\fboxrule}{1pt}%
\fcolorbox{subtitle}{background}{\ig[width=0.6\textwidth]{gamma_analysis.pdf}}
\end{center}
\begin{itemize}
\item When $\tau_X = \tau_\epsilon$: equivalent to an independent random sample
\item When $\tau_X = 1$: recover the classical result for longitudinal studies comparing treatment to control group
\end{itemize}

\end{frame}

\begin{frame}{Conclusions}
\bigskip

\begin{itemize}
\item I hope this shows good examples of exciting work \textbf{you} could do at UW
\item Please reach out to me with any questions after your visit
\item Special thanks to Professor Karl Broman for the template for these slides
\end{itemize}
\end{frame}



\end{document}